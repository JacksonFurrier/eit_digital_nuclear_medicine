\documentclass[aspectratio=169]{beamer}

\usepackage{amsmath,amssymb,amsfonts}
\usepackage{subcaption}
\usepackage{float}
\usepackage{cleveref}

\makeatletter
\@addtoreset{subfigure}{figure}
\makeatother


\usetheme{Execushares}


% Title page details: 
\title{Biomedical and nuclear imaging for entrepreneurs} 
\author{ 
    \textbf{Ádám István Szűcs}
}
\date{XXX XX, 2024}


\begin{document}

% Title page frame
\begin{frame}
    \titlepage 
\end{frame}

% Outline frame
\begin{frame}{Outline}
    \tableofcontents
\end{frame}

\section{Environment}
\subsection{Local -- Hungary}
\begin{frame}{CAD diagnosis with SPECT MPI}
    \begin{itemize}
        \item<1-> Radiopharmaceuticals injected intravenously
        \item<2-> Gamma radiation is captured by SPECT cameras
        \item<3-> Reconstruction with various methods
        \item<4-> Evaluation of segmented and reoriented left ventricles
    \end{itemize}
\end{frame}

\subsection{Global -- EU}
\begin{frame}{CAD diagnosis with SPECT MPI}
    \begin{itemize}
        \item<1-> Radiopharmaceuticals injected intravenously
        \item<2-> Gamma radiation is captured by SPECT cameras
        \item<3-> Reconstruction with various methods
        \item<4-> Evaluation of segmented and reoriented left ventricles
    \end{itemize}
\end{frame}

\subsection{Global -- US and the rest}
\begin{frame}{CAD diagnosis with SPECT MPI}
    \begin{itemize}
        \item<1-> Radiopharmaceuticals injected intravenously
        \item<2-> Gamma radiation is captured by SPECT cameras
        \item<3-> Reconstruction with various methods
        \item<4-> Evaluation of segmented and reoriented left ventricles
    \end{itemize}
\end{frame}

\section{Portfolio}
\subsection{Nuclear imaging}
\begin{frame}{Reorientation}
    \begin{itemize}
        \item<1->Manual reorientation is slow
        \item<2->Automatic reorientation is fast
        \item<3->Reorientation must be accurate
        \item<4->Automatic reorientation is easy if we have a segmented LV
    \end{itemize}
\end{frame}

\subsection{Microbiology imaging}

\begin{frame}{Automatic segmentation methods}
    \begin{center}
        \begin{tabular}{ c || c | c | c}
            Labeled data reqs. & Model based & Self-supervised & Supervised \\
            \hline
            No data & $\checkmark$ & $\times$ & $\times$ \\
            Small set & $\times$ & $\checkmark$ & $\times$ \\
            Large set & $\times$ & $\times$ & $\checkmark$
        \end{tabular}

        \vspace{2em}

        \begin{tabular}{ c || c | c | c}
            & Model based & Self-supervised & Supervised \\
            \hline
            Accuracy & $\times$ & ? & $\checkmark$ \\
            Robustness & $\times$ & ? & $\checkmark$ \\
        \end{tabular}
    \end{center}
\end{frame}

\section{Solution}
\subsection{Education -- courses}
\begin{frame}{Representation learning}
    \centering Standard 3D U-Net \cite{cciccek20163d}\\ with self supervised representation learning
\end{frame}

\subsection{Company network}
\begin{frame}{Heavy augmentation}
    Data augmentation
    \begin{itemize}
        \item affine transformation 
        \item elastic transformation
        \item extra noise
        \item jigsaw puzzle with maximal hamming distance, 1000 permutations
    \end{itemize}
    Training strategy as in 3D SSL U-Net with two-stage training as in \cite{adam2023}
\end{frame}


\section{Q\&A}
\begin{frame}
    \begin{center}
        \Huge \emph{Thank You}
    \end{center}

    \tiny The research was supported by the project No. 2019-1.3.1-KK-2019-00011 financed by the National Research, Development and Innovation Fund of Hungary under the Establishment of Competence Centers, Development of Research Infrastructure Programme funding scheme. 
\end{frame}

\backupbegin
\begin{frame}[c]
    \frametitle{The code}
    \centering
    Abrakadabra
\end{frame}

\begin{frame}%[allowframebreaks]{Outline}
    \frametitle{References}
    \bibliographystyle{apalike}
    {
    \tiny
    \bibliography{pitch.bib}
    }
\end{frame}
\backupend

\end{document}